\documentclass{article}
\usepackage[utf8]{inputenc}
\usepackage{listings}
\usepackage{xcolor}

% Define colors for code highlighting
\definecolor{codegreen}{rgb}{0,0.6,0}
\definecolor{codegray}{rgb}{0.5,0.5,0.5}
\definecolor{codepurple}{rgb}{0.58,0,0.82}
\definecolor{backcolour}{rgb}{0.95,0.95,0.92}

% Code style
\lstdefinestyle{mystyle}{
    backgroundcolor=\color{backcolour},   
    commentstyle=\color{codegreen},
    keywordstyle=\color{magenta},
    numberstyle=\tiny\color{codegray},
    stringstyle=\color{codepurple},
    basicstyle=\ttfamily\footnotesize,
    breakatwhitespace=false,         
    breaklines=true,                 
    captionpos=b,                    
    keepspaces=true,                 
    numbers=left,                    
    numbersep=5pt,                  
    showspaces=false,                
    showstringspaces=false,
    showtabs=false,                  
    tabsize=2
}
\lstset{style=mystyle}

\title{RPC File Transfer}
\author{Tran Hung Thinh (BI12-428)}
\date{\today}

\begin{document}

\maketitle

\section{Introduction}
The RPC File Transfer system uses \textbf{gRPC} to enable seamless file transfers between a client and a server. This document describes its design, functionality, and implementation in detail.

\section{Service Definition}

\subsection{Overview}
The RPC service is defined using \textbf{Protocol Buffers (protobuf)}. It includes the following operations:
\begin{itemize}
    \item \textbf{SendFile:} Transfers a file from the client to the server.
    \item \textbf{ReceiveFile:} Downloads a file from the server to the client in chunks.
\end{itemize}

\subsection{Message Types}
The service uses custom protobuf messages:
\begin{itemize}
    \item \textbf{FileRequest:} Contains the file name and file content.
    \item \textbf{FileResponse:} Indicates the success of a file transfer.
    \item \textbf{FileChunk:} Represents chunks of a file for efficient transfer.
    \item \textbf{Empty:} Used as a placeholder when no additional data is needed.
\end{itemize}

\section{System Organization}

\subsection{Server-Side}
The server is responsible for:
\begin{itemize}
    \item Listening for client requests.
    \item Handling file transfer operations using the \texttt{SendFile} and \texttt{ReceiveFile} methods.
    \item Writing received files to disk.
\end{itemize}

\subsection{Client-Side}
The client performs the following tasks:
\begin{itemize}
    \item Sending files to the server using the \texttt{SendFile} method.
    \item Receiving files from the server in chunks using the \texttt{ReceiveFile} method.
\end{itemize}

\section{Implementation Details}

\subsection{Protocol Buffer Definition}
The protobuf file defines the service and message structures:
\begin{lstlisting}[language=protobuf, caption=Protocol Buffer Definition]
syntax = "proto3";

package filetransfer;

service FileTransfer {
    rpc SendFile(FileRequest) returns (FileResponse);
    rpc ReceiveFile(FileChunk) returns (Empty);
}

message FileRequest {
    string filename = 1;
    bytes content = 2;
}

message FileResponse {
    bool success = 1;
}

message FileChunk {
    bytes content = 1;
}

message Empty {}
\end{lstlisting}

\subsection{Client Implementation}
The client implementation (\texttt{rpcclient.cpp}) includes:
\begin{lstlisting}[language=C++, caption=Client Implementation]
#include <iostream>
#include <fstream>
#include <string>
#include <grpcpp/grpcpp.h>
#include "file_transfer.grpc.pb.h"

using grpc::Channel;
using grpc::ClientContext;
using grpc::Status;
using filetransfer::FileTransfer;
using filetransfer::FileRequest;
using filetransfer::FileResponse;
using filetransfer::FileChunk;
using filetransfer::Empty;

class FileTransferClient {
public:
    FileTransferClient(std::shared_ptr<Channel> channel)
            : stub_(FileTransfer::NewStub(channel)) {}

    bool SendFile(const std::string& filename) {
        std::ifstream file(filename, std::ios::binary);
        if (!file.is_open()) {
            std::cerr << "Error opening file for reading" << std::endl;
            return false;
        }

        FileRequest request;
        request.set_filename(filename);
        std::string content((std::istreambuf_iterator<char>(file)), (std::istreambuf_iterator<char>()));
        request.set_content(content);

        FileResponse response;
        ClientContext context;
        Status status = stub_->SendFile(&context, request, &response);
        if (status.ok() && response.success()) {
            std::cout << "File sent successfully!" << std::endl;
            return true;
        } else {
            std::cerr << "Error sending file: " << status.error_message() << std::endl;
            return false;
        }
    }

    void ReceiveFile() {
        FileTransfer::Stub stub(grpc::CreateChannel("localhost:50051", grpc::InsecureChannelCredentials()));
        Empty request;
        FileChunk chunk;
        ClientContext context;

        std::ofstream file("received_file.txt", std::ios::binary);
        if (!file.is_open()) {
            std::cerr << "Error opening file for writing" << std::endl;
            return;
        }

        while (!file.eof()) {
            chunk.set_content(std::string((std::istreambuf_iterator<char>(file)), (std::istreambuf_iterator<char>())));
            Status status = stub.ReceiveFile(&context, chunk, &request);
            if (!status.ok()) {
                std::cerr << "Error receiving file: " << status.error_message() << std::endl;
                return;
            }
        }
        std::cout << "File received successfully!" << std::endl;
    }

private:
    std::unique_ptr<FileTransfer::Stub> stub_;
};

int main(int argc, char** argv) {
    FileTransferClient client(grpc::CreateChannel("localhost:50051", grpc::InsecureChannelCredentials()));
    client.SendFile("sample_file.txt");
    client.ReceiveFile();
    return 0;
}
\end{lstlisting}

\subsection{Server Implementation}
The server implementation (\texttt{rpcserver.cpp}) includes:
\begin{lstlisting}[language=C++, caption=Server Implementation]
#include <iostream>
#include <memory>
#include <string>
#include <fstream>
#include <grpcpp/grpcpp.h>
#include "file_transfer.grpc.pb.h"

using grpc::Server;
using grpc::ServerBuilder;
using grpc::ServerContext;
using grpc::Status;
using filetransfer::FileTransfer;
using filetransfer::FileRequest;
using filetransfer::FileResponse;
using filetransfer::FileChunk;
using filetransfer::Empty;

class FileTransferServiceImpl final : public FileTransfer::Service {

    Status SendFile(ServerContext* context, const FileRequest* request, FileResponse* response) override {

        std::ofstream file(request->filename(), std::ios::binary);
        if (!file.is_open()) {
            std::cerr << "Error opening file for writing" << std::endl;
            return Status::OK;
        }

        file.write(request->content().c_str(), request->content().length());
        file.close();

        response->set_success(true);
        return Status::OK;
    }

    Status ReceiveFile(ServerContext* context, const FileChunk* request, Empty* response) override {

        std::ofstream file("received_file.txt", std::ios::binary | std::ios::app);
        if (!file.is_open()) {
            std::cerr << "Error opening file for writing" << std::endl;
            return Status::OK;
        }

        file.write(request->content().c_str(), request->content().length());
        file.close();
        return Status::OK;
    }
};

void RunServer() {
    std::string server_address("0.0.0.0:50051");
    FileTransferServiceImpl service;

    ServerBuilder builder;
    builder.AddListeningPort(server_address, grpc::InsecureServerCredentials());
    builder.RegisterService(&service);

    std::unique_ptr<Server> server(builder.BuildAndStart());
    std::cout << "Server listening on " << server_address << std::endl;
    server->Wait();
}

int main() {
    RunServer();
    return 0;
}
\end{lstlisting}

\section{File Transfer Workflow}

\subsection{Sending a File}
\begin{enumerate}
    \item The client reads the file content.
    \item Constructs a \texttt{FileRequest} message.
    \item Sends the message to the server using \texttt{SendFile}.
    \item The server writes the file to disk and responds with a success status.
\end{enumerate}

\subsection{Receiving a File}
\begin{enumerate}
    \item The client requests a file in chunks.
    \item The server sends chunks using \texttt{FileChunk} messages.
    \item The client reconstructs the file locally.
\end{enumerate}

\section{Conclusion}
This RPC File Transfer system showcases efficient file handling using gRPC. The modular design ensures robust client-server communication. Future enhancements can include error handling and progress tracking.

\end{document}
